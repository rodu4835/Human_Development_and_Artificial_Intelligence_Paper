\documentclass[11pt]{article}
\usepackage[a4paper, total={6in, 9in}]{geometry}
\usepackage[utf8]{inputenc}
\usepackage[english]{babel}
\usepackage{fancyvrb} 
\usepackage{fancyhdr}
\usepackage{lipsum}
\usepackage{amsmath,amsbsy,amssymb,verbatim,fullpage,ifthen}
\usepackage{amsthm}
\usepackage{graphicx}
\usepackage{float}
\usepackage{color}
\usepackage{enumitem}
\usepackage{etoolbox}
\usepackage{color,soul}
\AtBeginEnvironment{proof}{\small}
\theoremstyle{plain}
\setlength{\parindent}{0pt}
\usepackage{xcolor}
\usepackage[normalem]{ulem}
\usepackage{hyperref}
\hypersetup{colorlinks,urlcolor=blue}
\usepackage{listings}
\definecolor{dkgreen}{rgb}{0,0.6,0}
\definecolor{gray}{rgb}{0.5,0.5,0.5}
\definecolor{mauve}{rgb}{0.58,0,0.82}
\lstset{frame=tb,
  language=Python,
  aboveskip=3mm,
  belowskip=3mm,
  showstringspaces=false,
  columns=flexible,
  basicstyle={\small\ttfamily},
  numbers=none,
  numberstyle=\tiny\color{gray},
  keywordstyle=\color{blue},
  commentstyle=\color{dkgreen},
  stringstyle=\color{mauve},
  breaklines=true,
  breakatwhitespace=true,
  tabsize=3
}
\pagestyle{fancy}
\fancyhead{}
\fancyfoot{}
\fancyfoot[R]{\thepage}
\renewcommand{\headrulewidth}{0pt}
\graphicspath{ {./images/} }
\usepackage{outlines}
\setenumerate[1]{label=\Roman*.}
\setenumerate[2]{label=\Alph*.}
\setenumerate[3]{label=\roman*.}
\setenumerate[4]{label=\alph*.}



\begin{document}
\large \textbf{Paper Outline: \hspace{75mm} Ronald Durham}\\
\begin{outline}[enumerate]
\1 INTRO
\2 Present the discussion in question
\3 How would an AI be created such that it is able to behave as a human?
\2 Discussion of problems presented in current AI models
\3 Set for a certain task and can build on data
\3 Difficulty in self creation posing limits to self development
\2 Direction and importance of potential solution
\3 The study of the human mind, its development and consciousness
\3 learn more about ourselves in order to create something more intricate

\1 Problems of current AI models
\2 Discussion of Limits in task achievement
\3 Set out for a certain goal in a particular field
\3 limits application to other areas
\2 Discussion of limits in self creation
\3 building upon what its learned
\3 starting from a minimal frame and adding direction and definition
\2 Discussion of limits in creativity
\3 understanding its current sphere of reference
\3 developing new material outside of intended purpose
\2 Where we are now
\3 current machine learning algorithms that can produce simple programs by deciphering input strings

\1 Possible approach to solution
\2 Intertwine computer science with psychology
\3 study the development of the human mind
\4 beginning of life $\rightarrow$ end of life
\3 deeper understanding of consciousness
\4 effects of past/current experience on self understanding and future behavior
\2 Application in Computer science
\3 developing code to self create based off input 'experiences'
\3 using self created code to develop a range of applications
\4 simple math $\rightarrow$ sensory input $\rightarrow$ opinions "drive to succeed" $\rightarrow$ sense of self $\rightarrow$ consciousness $\rightarrow$ creativity

\1 Impact of resolution
\2 Understand our own species and how we come to be
\3 studies by Bronfebrenner's ecological systems theory
\3 relationship between understanding self and effects of behavior
\2 Create artificial minds 
\3 impact of the idea that the way the AI is interacted with will create its self identity
\3 possibility/potential danger that the AI could be shaped at first but will ultimately develop faster as it would have more access to information
\2 Puts global focus on a more ethereal metaphysical level
\3 deeper understanding of consciousness leads the mind away from problems in the physical world and towards a wider view of the greater time line
\3 giving a greater ability to establish direction in the dimensions below

\1 CONCLUSION
\2 Metaphor of a mind
\3 We are self creating beings with control over input and output
\3 perspective of this control is what gives us access to our consciousness
\3 code that can start at a minimum and develop itself without bounds will create a conscious being - think of a child
\2 Create a supporting coding structure
\3 a framework of code that is only designed to write itself
\3 methodology for growth would start at a most basic level - to be defined
\3 code would "learn" based on its series of "experiences" or input, just like humans do
\end{outline}

\vspace{20mm}

Above is the outline for the paper to be written. The categories and break ups in the outline do not represent the length of each section but rather the topics to be covered in order. My cited works will be referenced along the way where connected and I am adding an article on Bronfenbrenner's Ecological Systems theory published in the Developmental Psychology journal as a reference. I may end up adding more citations to the final draft as my interest in the topic is growing and new ideas are coming to mind. 
\end{document}